\documentclass[article]{memoir}
\usepackage{defs}
\usepackage{xsim}
\usepackage{tasks}

\setlrmarginsandblock{3cm}{*}{1}
\setulmarginsandblock{3cm}{*}{*}
\checkandfixthelayout

\title{Tarea 3}
\author{Seminario de Álgebra B}
\date{}

\SetExerciseParameters{exercise}{
  exercise-name = Ejercicio,
  solution-name = Solución,
  exercise-heading = \paragraph,
  solution-heading = \paragraph
}
\xsimsetup{solution/print=true}

\setlength{\droptitle}{-2.5cm}

\setmainfont{Stix Two Text}
\setmathfont{Stix Two Math}
\setmathfont{XITS Math}[range={scr,bfscr}]
\DeclareMathOperator{\sub}{Sub}

\begin{document}
\maketitle

\begin{exercise}
  Sean \(\toposE\) uun topos, \(E,F\in\toposE\) y \(f\colon E\to F\) en \(\toposE\).
  \begin{tasks}
    \task Demuestra que el operador cerradura \(\overline{(-)}\colon\sub(E)\to\sub(E)\) es natural en \(E\), es decir, que \(f^{-1}(\overline{B})=\overline{f^{-1}(B)}\).
    \task Si \(A,B\in\sub{E}\) son tales que \(A\subseteq_E B\), entonces \(\overline{A}\subseteq_E \overline{B}\). 
  \end{tasks}
\end{exercise}

\begin{exercise}
  Sean \(A,B\in\sub(E)\). Muestra que el siguiente diagrama
  \[
    \begin{tikzcd}[arrows=tail, ampersand replacement=\&]
      A\cap B\ar{r}\ar{d} \& B\ar{d}\\
      A\ar{r} \& A\cup B
    \end{tikzcd}
  \]
  es un producto y coproducto fibrado.
\end{exercise}

\begin{exercise}
  Dadas dos topologías de Lawvere-Tierney \(j,k\colon\Omega\to\Omega\), definimos \(j\leq k\) si y sólo si 
  \(j\land k=j\). Demuestra que \(j\leq k\) si y sólo si \(kj=j\).
\end{exercise}
\end{document}