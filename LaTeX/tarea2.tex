\documentclass[article]{memoir}
\usepackage{defs}
\usepackage{xsim}
\usepackage{tasks}

\setlrmarginsandblock{3cm}{*}{1}
\setulmarginsandblock{3cm}{*}{*}
\checkandfixthelayout

\title{Tarea 2}
\author{Seminario de Álgebra B}
\date{}

\SetExerciseParameters{exercise}{
  exercise-name = Ejercicio,
  solution-name = Solución,
  exercise-heading = \paragraph,
  solution-heading = \paragraph
}
\xsimsetup{solution/print=true}

\setlength{\droptitle}{-2.5cm}

\setmainfont{Stix Two Text}
\setmathfont{Stix Two Math}
\setmathfont{XITS Math}[range={scr,bfscr}]
\DeclareMathOperator{\sub}{Sub}

\begin{document}
\maketitle

\begin{exercise}
  Sea \(C\in\toposE\). Muestra que \(C^C\) es un objeto monoide en \(\toposE\). Esto es, existen flechas
  \(e\colon 1\to C^C\) y \(m\colon C^C\times C^C\to C^C\) tales que los siguientes diagramas conmutan
  \[
    \begin{tikzcd}[ampersand replacement=\&]
      C^C\times C^C\times C^C\ar{r}{\id\times m}\ar{d}[swap]{m\times\id} \& C^C\times C^C\ar{d}{m}\\
      C^C\times C^C\ar{r}[swap]{m} \& C^C
    \end{tikzcd}
    \quad
    \begin{tikzcd}[ampersand replacement=\&]
      1\times C^C\ar{r}{e\times\id}\ar{rd}[swap]{p_{C^C}} \& C^C\times C^C\ar{d}{m} 
      \& C^C\times 1\ar{l}[swap]{\id\times e}\ar{ld}{p_{C^C}}\\
      \& C^C
    \end{tikzcd}
  \]
\end{exercise}

\begin{exercise}
  Demuestra que la biyección \(\toposE(A,\Omega)\cong\sub_{\toposE}(A)\) es natural en \(A\).
\end{exercise}

\begin{exercise}
  Si \(f\colon\Omega\to\Omega\) es un mono, entonces \(ff=\id_{\Omega}\).
\end{exercise}

\begin{exercise}
  El par núcleo de una flecha \(f\colon A\to B\) consta de dos flechas \(a,b\colon R\to A\) tales que el diagrama
  \[
    \begin{tikzcd}[ampersand replacement=\&]
      R\ar{r}{a}\ar{d}[swap]{b} \& A\ar{d}{f}\\
      A\ar{r}[swap]{f} \& B
    \end{tikzcd}
  \]
  es un producto fibrado. Demuestra que el par núcleo cumple lo siguiente:
  \begin{tasks}
    \task la flecha \((a,b)\colon R\to A\times A\) es mono,
    \task la diagonal \(\Delta_A\colon A\to A\times A\) está contenida en \((a,b)\), es decir, existe \(\rho\colon A\to R\) tal que el siguiente diagrama conmuta
    \[
      \begin{tikzcd}[ampersand replacement=\&]
        A\ar[dashed]{rr}{\rho}\ar{rd}[swap]{\Delta_A} \&\& R\ar{ld}{{(a,b)}}\\
        \& A\times A
      \end{tikzcd}
    \]
    \task \((b,a)\) está contenida en \((a,b)\), es decir, existe una flecha \(\sigma\colon R\to R\) que hace conmutar al diagrama
    \[
      \begin{tikzcd}[ampersand replacement=\&]
        R\ar[dashed]{rr}{\sigma}\ar{rd}[swap]{{(b,a)}} \&\& R\ar{ld}{{(a,b)}}\\
        \& A\times A
      \end{tikzcd}
    \]
    \task Si consideramos el producto fibrado de abajo a la izquierda, entonces existe una flecha \(\tau\colon T\to R\) tal que el diagrama de abajo a la izquierda conmuta
    \[
      \begin{tikzcd}[ampersand replacement=\&]
        T\ar{r}{q}\ar{d}[swap]{p} \& R\ar{d}{a}\\
        R\ar{r}[swap]{b} \& A
      \end{tikzcd}
      \qquad
      \begin{tikzcd}[ampersand replacement=\&]
        T\ar[dashed]{rr}{\tau}\ar{rd}[swap]{{(ap,bq)}} \&\& R\ar{ld}{{(a,b)}}\\
        \& A\times A\mathrlap{.}
      \end{tikzcd}
    \]
  \end{tasks}
\end{exercise}
\end{document}