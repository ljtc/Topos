\documentclass[article]{memoir}
\usepackage{defs}
\usepackage{xsim}
\usepackage{tasks}

\title{Tarea 2}
\author{Seminario de Álgebra B}
\date{}

\SetExerciseParameters{exercise}{
  exercise-name = Ejercicio,
  solution-name = Solución,
  exercise-heading = \paragraph,
  solution-heading = \paragraph
}

\setlength{\droptitle}{-3.5cm}

\DeclareMathOperator{\sub}{Sub}

\begin{document}
\maketitle

\begin{exercise}
  Sea \(C\in\toposE\). Muestra que \(C^C\) es un objeto monoide en \(\toposE\). Esto es, existen flechas
  \(e\colon 1\to C^C\) y \(m\colon C^C\times C^C\to C^C\) tales que los siguientes diagramas conmutan
  \[
    \begin{tikzcd}[ampersand replacement=\&]
      C^C\times C^C\times C^C\ar{r}{\id\times m}\ar{d}[swap]{m\times\id} \& C^C\times C^C\ar{d}{m}\\
      C^C\times C^C\ar{r}[swap]{m} \& C^C
    \end{tikzcd}
    \quad
    \begin{tikzcd}[ampersand replacement=\&]
      1\times C^C\ar{r}{e\times\id}\ar{rd}[swap]{p_{C^C}} \& C^C\times C^C\ar{d}{m} 
      \& C^C\times 1\ar{l}[swap]{\id\times e}\ar{ld}{p_{C^C}}\\
      \& C^C
    \end{tikzcd}
  \]
\end{exercise}

\begin{exercise}
  Demuestra que la biyección \(\toposE(A,\Omega)\cong\sub_{\toposE}(A)\) es natural en \(A\).
\end{exercise}

\begin{exercise}
  Muestra que \(\con\op\) es equivalente a la categoría de álgebras de Boole completas y atómicas, \(\boo\).
\end{exercise}
\end{document}