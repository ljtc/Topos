\documentclass[article]{memoir}
\usepackage{defs}
\usepackage{xsim}
\usepackage{tasks}

\title{Tarea 1}
\author{Seminario de Álgebra B}
\date{}

\begin{document}
\maketitle

\begin{exercise}
Sean \(\catA\) y \(\catB\) dos categorías y \(F\colon\catA\to\catB\) y \(G\colon\catB\to\catA\) dos funtores. Si 
\(F\dashv G\), entonces demuestra que la counidad \(\varepsilon\colon FG\to \Id_{\catB}\) es una transformación natural y que para cada \(B\in\catB\) la componente \(\varepsilon_B\colon FGB\to B\) es una flecha universal de \(F\) en \(B\), es decir, que el siguiente diagrama conmuta
\[
  \begin{tikzcd}[ampersand replacement=\&]
    FGB\ar{r}{\varepsilon_B} \& B \&\& GB\\
    FA\ar{ru}[swap]{\forall f}\ar[dashed]{u}{Fg} \&\&\& A\mathrlap{.}\ar{u}{\exists !}[swap]{g}
  \end{tikzcd}
\]
\end{exercise}

\begin{exercise}
  Dado un orden parcial \((P,\leq)\) podemos formar una categoría \(\catP\) cuyos objetos son los elementos de \(P\) y hay una flecha \(p\to q\) si y sólo si \(p\leq q\). Ahora considera dos órdenes parciales vistos como categorías, \(\catP\) y \(\catQ\).
  \begin{tasks}
    \task ¿Qué es funtor \(F\colon\catP\to\catQ\)?
    \task Si \(F,G\colon\catP\to\catQ\) son funtores, entonces ¿que se debe satisfacer para que haya una transformación natural \(\tau\colon F\to G\)?
    \task Si \(F\colon\catP\to\catQ\) y \(G\colon\catQ\to\catP\) son funtores, entonces ¿que condición se debe satisfacer para que \(F\dashv G\)? \textit{Sugerencia: en este caso lo más fácil es describir la biyección de flechas}.
  \end{tasks}
\end{exercise}


\end{document}